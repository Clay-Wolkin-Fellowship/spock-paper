\documentclass[conference]{IEEEtran}

\usepackage{graphicx}
\usepackage[section]{placeins}
\newcommand{\comment}[1]{}

\begin{document}

\title{Qualitative Cache Performance Analysis}

\author{}
%\author{Andrew Carter \and Max Korbel \and Paula Ning \and Seth Pugsley \and Josef Spjut}

\maketitle

\begin{abstract}
The effectiveness of caching policies has been measured by a number of metrics. 
The ultimate quantitative measure is overall system performance.
Other metrics such as hit rate, misses per thousand instructions and
instructions per cycle are also regularly used in the literature to
compare cache behaviors.
In this work we propose a novel class of metrics based on the idea
that any memory element should ideally be kept out of the cache
for as long as possible before being fetched again, an idea inspired
by Belady's algorithm.
In general we measure the Time to Recache for each evicted element as a qualitative measurement of a caching policy's performance.
Among this set of metrics, we analyze the number of Memory Accesses and Memory Misses to Recache, as well as a rough approximation
of Wall-Time to Recache.
We believe that these metrics will not only be useful for comparing the performance of two different caching policies,
  but allow designers of such policies to indentify memory access
  patterns that are problematic for existing policies.
We also provide our simulation and testing methodology and source to
assist others in applying these metrics to their own
studies.
\end{abstract}

\section{Introduction}
Caching is used in processor design to keep data close to the computational units.
This reduces memory access latency, which in turn reduces the cycles per instruction (CPI) of a program, ideally improving system performance.
In an effort to improve the effectiveness of caches, a variety of cache policies have been proposed.
Laszlo Belady~\cite{belady66} proposed an optimal algorithm that involves always evicts the memory that will be used furthest in the future.
Unfortunately this requires knowing ahead of time what memory the program will access and the order in which that memory will be accessed.
Online algorithms have been developed to approximate Belady's algorithms,
	these include Least Recently Used (LRU), Not Recently Used (NRU),
	and more recently Re-reference Interval Predicition (RRIP)~\cite{jaleeltheobald10},
	in both Static RRIP (SRRIP) and Dynamic RRIP (DRRIP).
LRU and NRU were proposed on the basis that if a processor just used a cache block,
	it is likely to be used sooner than a cache block it has not used in a while.
The RRIP policies attempt to propose the future usefulness of cache blocks,
	and evicting cache blocks that are not likely to be used in the future.
These policies are discussed at length in Section~\ref{sec:policies}

In this paper we propose the use of a class of Time to Recache (TTR) metrics in offering insight into why different cache management policies perform better than others.  
There are a variety of different ways to measure TTR, including Memory Accesses to Recache (MATR), Memory Misses to Reacache (MMTR), Wall-time to Recache (WTTR). 
This class of metrics refers to the time spent by a cache line after it has been evicted from the cache and before being fetched again.  
It is related to, but distinct from, the notion of ``reuse distance,'' which refers to the amount of time between successive accesses to a given cache block.
The TTR-based metrics are discussed in detail in Section~\ref{sec:metrics}.


\section{System Performance Metrics}

Computer performance is dependent on a variety of factors, ranging
from very low level features like the latency of individual functional
units, bypass networks,
and out-of-order hardware, up through very high level features such as
operating systems, disk I/O, and network latency.  In this paper we
are focusing our attention on the performance of the LLC.  The quality
of an LLC is typically gauged by two factors, the improvement in
performance, measured in Instructions Per Clock (IPC),
that it affords to the processor it backs, and the reduction of Misses
Per
1000 Instructions (MPKI), which equates to a reduction in the
number of long latency DRAM main memory accesses.

\subsection{IPC}

The number of instructions a CPU can complete in a single clock cycle
can be equated with its absolute performance.  If a processor can
complete more instructions in a given clock cycle than another,
its performance is better.  Hardware caches play a critical role in
boosting this number.  The closer that data sits to the functional
units, the higher performance can be.  In a typical three level cache
hierarchy, it can take 10x longer to access the third level of
cache than the first, and another 10x longer to access main memory.
Finding data as close to the processor as possible is critical for
high performance.

\subsection{MPKI}

One of the LLC's main jobs is to redue the number of DRAM accesses
that are
performed.  Each DRAM read access includes occupying a memory
controller's read buffer, waiting for this access's turn, and then
sending that read request across long wires to distant DRAM chips,
activating those DRAM chips, and then finally sending the data back to
the memory controller over several more cycles.  This description of
events omits what happens to the data after it gets back to the memory
controller, and also ignores the consequences of needing to write
dirty data from the LLC back to the cache.  There is a lot of work
that has to be done in the event of an LLC miss, so reducing the MPKI
of a cache is a popular and important metric to look at.

%\subsection{More of the Story}
% This needs work
%
%Measuring IPC and MPKI can tell you about how well your LLC is
%performing, but they don't tell you about what's going on inside the
%LLC to provide the performance.  Looking at IPC alone does not tell
%you anything about where to attribute the increase in performance.
%Similarly, looking at MPKI alone does not tell you about the
%underlying memory access patterns that caused that MPKI.  In this
%paper, we look at the Time to Recache (TTR) metric in order to get a
%more complete view of the behavior of an LLC.

\subsection{TTR}
\label{sec:metrics}
TODO: Update with WTTR, MATR, MMTR etc.

In this work we propose the Time to Recache (TTR) methodology for
examining the behavior and effectiveness of the LLC.  TTR is defined
as the amount of time (measured in a variety of different manners) that a cache
block spends outside of the cache after it has been evicted and
before it is accessed again.  Note that this is distinct from the
concept of reuse distance.  Reuse distance is the time between
successive accesses to a piece of data or cache block.  TTR doesn't
take into account the amount of time a cache block spent in the LLC
before it was evicted.  TTR is only concerned with the time spent
after eviction and before reuse.

We came up with three possible TTR metrics. Wall Time to Recache (WTTR), measures
the amount of time the CPU would take (either in cycles, or in seconds) before a cache block is accessed again.
In this paper we approximate WTTR by assigning a cost to instructions and cache misses.
The remaining two metrics are Memory Accesses to Recache (MATR) and Memory Misses to Recache (MMTR).
Both of these count memory instructions in between an eviction and a recache, but MMTR only counts misses.
We will show that these metrics have different values.

Belady's optimal algorithm~\cite{belady66} for cache eviction always
evicts
the cache block whose reuse is furthest in the future, allowing that
free space to be used as long as possible by other data before being
recached.  It is impossible to know at runtime for general workloads
which cache block has the furthest reuse distance, hence why there are
so many different caching policies that use various heuristics in an
effort to approach the effectiveness of this optimal algorithm.

Measuring the IPC and MPKI of a workload using one caching policy, and
comparing that to the IPC and MPKI of running that workload with a
different caching policy can give you some sense of how close each of
those caching policy comes to the optimal solution.  This is,
however, and indirect approach to quantifying how well a caching
policy is performing.  Tracking the TTR is a direct means of
comparing two caching policies.

TTR is an effective metric because it asks the question every time a
cache block is brought into the cache, ``have I seen this block
before, and if so, how long ago was it?''  If caching policy A answers
this question with ``4000 cycles ago,'' and caching policy B answers
this question with ``6000 cycles ago,'' then caching policy B has done
a better job at evicting that cache block early and allowing that
space to be used by other data.  With TTR, a higher number is better.
A low TTR number means that the cache block was evicted and then
recached very soon afterwards, suggesting that it should not have
been evicted in the first place.

However not one TTR metric tells the whole story.
In aggregate, it is hard to tell the difference between recaching after CPU intensive calculations, a small working set, and a large number of evictions.
By having all three metrics, we can perhaps determine that a low WTTR actually had a relatively high MMTR, indicating that the working set was simply too large for our cache.
Thus despite a low WTTR, it is probably hard to improve the cache over that range.
Similarly if MATR or MMTR is very low, but WTTR is very high, then that portion of code is probably CPU bound, and there is no need to improve the caching algorithm over that region.

In FIGURE HERE, we see an example of a TTR graph.  A TTR graph is a
line-graph representation of a histogram of TTR values.  The majority
of
the results in this paper are presented in this format.  The bins of a
TTR graph, along the X-axis,  are 10,000 cycle-long periods of time
that a cache block has
been absent from the cache before returning.  The Y-axis of the graph
is the number of cache blocks that were absent from the cache for that
long before returning.  For example, if 10 cache blocks had been
recached after being absent from the cache for 40,000 cycles each,
then the 4th bin of the graph would have the value of 10.

A TTR graph shows the distribution of how long cache blocks were
absent.  The intuition of how to read a TTR graph is as follows.  A
high Y value is generally bad, because it means there were many
evictions
and recaches.  Similarly, recaches that happen at a low TTR bin number
(low X value in the graph) are considered bad because it means that
when a cache block was evicted it was soon recached, suggesting that
cache block should not have been evicted in the first place.  Many TTR
graphs include humps in their distribution.  Humps that appear at low
TTR values are generally worse than humps that appear at high TTR
values, although the magnitude of the hump must also be considered.
High magnitude humps are generally worse than low magnitude humps,
although the position of these humps must also be considered.

The particular TTR graph in FIGURE HERE is comprised of four TTR
graphs positioned one above another, each of them representing one of
four different runs of the same benchmark, but using different caching
policies.  TODO FINISH THIS

\newcommand{\SAMPN}{100}
\newcommand{\SAMPK}{2000000}
\newcommand{\SAMP}{1400000}
\newcommand{\WARM}{100000}
\newcommand{\COOL}{500000}
\newcommand{\comment}[1]{}

\section{Methodology}
We simulate a set associative cache using a number of different
replacement policies.
We use a statistical sampling technique to keep our total simulation
time down while still ensuring our results are representative of the
overall behavior for the application in question.

Mention PIN, NPB, Custom Python Cache Simulation

\subsection{Sampling Technique}
We sampled the file to minimize the data set we operated on, while maintaining the integrity with results.
	\comment{Something about pin here}
	We took \SAMPN samples of \SAMPK memory access.
	By the central limit theorem we can approximate samples of this size as having a gaussian distribution.
	Each sample was divided into 3 sections, a warm-up period of \WARM memory accesses,
		a sampling period of \SAMP memory accesses, and a cool-down period of \COOL memory accesses.
	For each metric we warmed up the cache during the warm-up period,
		then measured how long it took for any cache line evicted during the sampling period to be recached.
	Cache lines that took longer than the cool-down period to recache were ignored.
	We felt that cache lines that took more than \COOL memory accesses to recache were unlikely to be interesting.

\subsection{Caching Policies}
\label{sec:policies}
BELADY

NRU

DRRIP

\section{Results}
We present results from Belady, Random(rand), FIFO, and all of the RRIP policies (SRRIP, DRRIP, BRRIP).
These polcies were run against a trace of sp\_omp run on a single core, with a 8-way set associative cache, with 7 bits of direct mapping within each set, and 64 byte cache lines.
We limited the graphs to only show MATR because we believe that for this particular trace,
 and this set of replacement polcies, this TTR best shows the usefulness of the TTR metrics in general.

Figures~\ref{matr:buc:belady}-\ref{matr:buc:brrip} show each individual replacement policy,
 as well as the error in measurement due to our sampling strategy.
Figures~\ref{matr:buc:belady:srrip}-\ref{matr:buc:drrip:brrip} show a side-by-side comparison of two different replacement policies.

In each graph, any $(x,y)$ point corresponds to $y$ evictions that took $x$ thousand memory accesses to be recached.
To smooth out the graph, these points have been collected into $1000$ buckets,
 each bucket represents an area of 500 Memory Access.
 
\FloatBarrier

\newcommand{\mkfigure}[4]{
\begin{figure}
\begin{center}
\includegraphics[width={0.9\columnwidth}]{{img/sp_omp.#2}.png}
\end{center}
\caption{Bucketed \uppercase{#1} for #3}
\label{#1:buc:#4}
\end{figure}
}

\newcommand{\mksfigure}[2]{\mkfigure{#1}{#2}{\uppercase{#2}}{#2}}
\newcommand{\mkvsfigure}[3]{\mkfigure{#1}{#2-#3}{\uppercase{#2} vs. \uppercase{#3}}{#2:#3}}
\newcommand{\mkallfigures}[1]{
\mksfigure{#1}{belady}
\mksfigure{#1}{rand}
\mksfigure{#1}{fifo}
\mksfigure{#1}{srrip}
\mksfigure{#1}{drrip}
\mksfigure{#1}{brrip}
\mkvsfigure{#1}{belady}{srrip}
\mkvsfigure{#1}{belady}{rand}
\mkvsfigure{#1}{belady}{drrip}
\mkvsfigure{#1}{srrip}{rand}
\mkvsfigure{#1}{srrip}{drrip}
\mkvsfigure{#1}{srrip}{fifo}
\mkvsfigure{#1}{drrip}{brrip}
}
\mkallfigures{matr}


\section{Results}

In this section we show the results of TTR tracking for our different
benchmarks and cache management policies.  We also include IPC and
MPKI numbers for comparison.  As stated
earlier, the purpose of this paper is not to showcase the relative
performance of one caching policy over another, but is rather to show
the effectiveness of the TTR visualization method at giving some
additional insight into system performance.
Furthermore, no attempt was made to ensure that the simulation windows
we used in gathering these statistics is necessarily representative of
program execution overall.  For this reason, our performance numbers
for the cache policies considered here may differ from previously
published
results for those cache policies.  Our goal is merely to show that TTR
visualization is a useful way to explain the performance we observed.

Our results are presented by showing the TTR graphs for LRU, NRU,
SRRIP, and DRRIP for each of the benchmarks, using both 4~MB and
8~MB.  The third graph shows the IPC and MPKI for each of those runs,
again separated by the two different cache sizes.  We have organized
the figures this way to make it easier to compare the shapes of the
TTR visualizations with the observed IPCs and MPKIs for each
benchmark.  When looking at the TTR visualizations for a given
benchmark and a given cache size, the scale for all four component
graphs (LRU, NRU, SRRIP, and DRRIP) is the same in both X and Y
directions, so they are all directly comparable.  When moving between
the 4~MB and 8~MB visualizatoins, the scales may be different.

****RESULTS GO HERE****

\section{Related Work}
Creating good cache line replacement policies is a well
studied area of computer architecture, and several recent papers have
worked on this problem.  In general it is the goal of every cache
policy design to keep around truly reused data while evicting data
that will not be reused for a very long time, but we mention here as
related work the
studies whose approach explicitly looks at reuse distance or some
variation of it as a part of its design or motivation.

Basu et. al.~\cite{basukirman07} investigate the concept of eviction-use distance
as motivation for their Scavenger LLC architecture, which identifies
cache blocks that are recently missed in the LLC, and then puts them
into a separate region of cache that protects them from their frequent
eviction.  In their motivation for this work,
mentions large Eviction-Use distance as one of the major contributors
to the problem and presents static values for Eviction-Use distance
for several benchmarks.  This differs from TTR visualization in that
they distilled the entire phenomenon down into a single value, and
TTR visualization shows the spectrum of all eviction-use distances.

Keramidas et. al.~\cite{keramidaspetoumenos07}, seek to genuinely predict the reuse
distance of cache blocks by using the program counter (PC) of a memory
access to index into a predictor structure, which is updated when a
reuse has been detected.  TTR visualization does not take the PC of
memory operations into account when plotting out recaching time,
although it could be interesting for future work to look at the TTR
graphs for each individual memory operation PC in a program.

Manikantan et. al.~\cite{manikantanrajan11}, identify delinquent memory operation PCs
and track
histograms of the next-use of the cache blocks brought in by them.
Some of the ways of the LLC are dedicated to blocks brought in by the
delinquent PCs, so that they do not pollute the rest of the ways.
This work only tracks reuse within a few dozen LLC misses of when a
block is last accessed, so their notion of reuse is much more
resitricted than that in TTR visualization.


\section{Conclusions}
In this work we have presented a class of metrics for cache studies
that provide deeper insight into the behavior of the replacement
algorithm when used for a particular application.
When Belady's algorithm is used, one can discover inflection points
that cause poor cache performance in the application.
A comparison between TTR for Belady's algorithm and any set of
replacement policies can provide insight for cache designers to choose
replacement policies that are appropriate for the class of
applications that the designer deems important.
Furthermore, our sampling technique allows for quick iteration in
policy design with rapid feedback to the developer.
MATR was reasonably straightforward to capture in our simulations, but
other types of TTR may be more useful or easy to record in other
contexts.

While we have presented these metrics in the context of CPU caches,
the potential application for these techniques extends beyond CPU
cache design.
For example, large web applications employ DRAM as a cache for large
databases held on arrays of disc drives as a method for increasing
input and output operation throughput.
Additionally, other application specific integrated circuits and
processors that utilize caching may find TTR useful.
In order to assist others in using our techniques, we have released
our source code publicly (reference withheld) and encourage revisions
and additions as appropriate for other domains.


\bibliographystyle{abbrv}
\bibliography{bib/main}

\end{document}
