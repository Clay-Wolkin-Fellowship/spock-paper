\section{System Performance Metrics}

Computer performance is dependent on a variety of factors, ranging
from very low level features like the latency of individual functional
units, bypass networks,
and out-of-order hardware, up through very high level features such as
operating systems, disk I/O, and network latency.  In this paper we
are focusing our attention on the performance of the LLC.  The quality
of an LLC is typically gauged by two factors, the improvement in
performance, measured in Instructions Per Clock (IPC),
that it affords to the processor it backs, and the reduction of Misses
Per
1000 Instructions (MPKI), which equates to a reduction in the
number of long latency DRAM main memory accesses.

\subsection{IPC}

The number of instructions a CPU can complete in a single clock cycle
can be equated with its absolute performance.  If a processor can
complete more instructions in a given clock cycle than another,
its performance is better.  Hardware caches play a critical role in
boosting this number.  The closer that data sits to the functional
units, the higher performance can be.  In a typical three level cache
hierarchy, it can take 10x longer to access the third level of
cache than the first, and another 10x longer to access main memory.
Finding data as close to the processor as possible is critical for
high performance.

\subsection{MPKI}

One of the LLC's main jobs is to redue the number of DRAM accesses
that are
performed.  Each DRAM read access includes occupying a memory
controller's read buffer, waiting for this access's turn, and then
sending that read request across long wires to distant DRAM chips,
activating those DRAM chips, and then finally sending the data back to
the memory controller over several more cycles.  This description of
events omits what happens to the data after it gets back to the memory
controller, and also ignores the consequences of needing to write
dirty data from the LLC back to the cache.  There is a lot of work
that has to be done in the event of an LLC miss, so reducing the MPKI
of a cache is a popular and important metric to look at.

%\subsection{More of the Story}
% This needs work
%
%Measuring IPC and MPKI can tell you about how well your LLC is
%performing, but they don't tell you about what's going on inside the
%LLC to provide the performance.  Looking at IPC alone does not tell
%you anything about where to attribute the increase in performance.
%Similarly, looking at MPKI alone does not tell you about the
%underlying memory access patterns that caused that MPKI.  In this
%paper, we look at the Time to Recache (TTR) metric in order to get a
%more complete view of the behavior of an LLC.

\subsection{TTR}
\label{sec:metrics}
TODO: Update with WTTR, MATR, MMTR etc.

In this work we propose the Time to Recache (TTR) methodology for
examining the behavior and effectiveness of the LLC.  TTR is defined
as the amount of time (measured in cycles or seconds) that a cache
block spends outside of the cache after it has been evicted and
before it is accessed again.  Note that this is distinct from the
concept of reuse distance.  Reuse distance is the time between
successive accesses to a piece of data or cache block.  TTR doesn't
take into account the amount of time a cache block spent in the LLC
before it was evicted.  TTR is only concerned with the time spent
after eviction and before reuse.

Belady's optimal algorithm~\cite{belady66} for cache eviction always
evicts
the cache block whose reuse is furthest in the future, allowing that
free space to be used as long as possible by other data before being
recached.  It is impossible to know at runtime for general workloads
which cache block has the furthest reuse distance, hence why there are
so many different caching policies that use various heuristics in an
effort to approach the effectiveness of this optimal algorithm.

Measuring the IPC and MPKI of a workload using one caching policy, and
comparing that to the IPC and MPKI of running that workload with a
different caching policy can give you some sense of how close each of
those caching policy comes to the optimal solution.  This is,
however, and indirect approach to quantifying how well a caching
policy is performing.  Tracking the TTR is a direct means of
comparing two caching policies.

TTR is an effective metric because it asks the question every time a
cache block is brought into the cache, ``have I seen this block
before, and if so, how long ago was it?''  If caching policy A answers
this question with ``4000 cycles ago,'' and caching policy B answers
this question with ``6000 cycles ago,'' then caching policy B has done
a better job at evicting that cache block early and allowing that
space to be used by other data.  With TTR, a higher number is better.
A low TTR number means that the cache block was evicted and then
recached very soon afterwards, suggesting that it should not have
been evicted in the first place.

In FIGURE HERE, we see an example of a TTR graph.  A TTR graph is a
line-graph representation of a histogram of TTR values.  The majority
of
the results in this paper are presented in this format.  The bins of a
TTR graph, along the X-axis,  are 10,000 cycle-long periods of time
that a cache block has
been absent from the cache before returning.  The Y-axis of the graph
is the number of cache blocks that were absent from the cache for that
long before returning.  For example, if 10 cache blocks had been
recached after being absent from the cache for 40,000 cycles each,
then the 4th bin of the graph would have the value of 10.

A TTR graph shows the distribution of how long cache blocks were
absent.  The intuition of how to read a TTR graph is as follows.  A
high Y value is generally bad, because it means there were many
evictions
and recaches.  Similarly, recaches that happen at a low TTR bin number
(low X value in the graph) are considered bad because it means that
when a cache block was evicted it was soon recached, suggesting that
cache block should not have been evicted in the first place.  Many TTR
graphs include humps in their distribution.  Humps that appear at low
TTR values are generally worse than humps that appear at high TTR
values, although the magnitude of the hump must also be considered.
High magnitude humps are generally worse than low magnitude humps,
although the position of these humps must also be considered.

The particular TTR graph in FIGURE HERE is comprised of four TTR
graphs positioned one above another, each of them representing one of
four different runs of the same benchmark, but using different caching
policies.  TODO FINISH THIS
